\chapter{Zielbestimmungen}

RION ist ein Package-Manager zum suchen, installieren, aktualisieren und verwalten von Packages für das Process Design Kit (PDK). Die Pakete für RION werden von der Serverseite INOR verwaltet.

\section{Problemstellung}
Zur Zeit werden Packages manuell von einem Webinterface heruntergeladen. Sie werden manuell installiert und aktualisiert. Dieser jetzige Weg ist jedoch unübersichtlich, fehleranfällig, inkonsistent, kostet sehr viel Zeit und erfordert sehr viel manuelle Arbeit, sowohl server- als auch clientseitig. Daher soll dieser Prozess durch Verwendung von RION, clientseitig für X-Fab-Kunden, und INOR, serverseitig für X-FAB-Entwickler, vereinfacht und automatisiert werden.

\section{Musskriterien}
\subsection{Funktionalität}
\begin{itemize}
		\item INOR muss die Packages serverseitig über eine Datenbank in Form von Archiven zur Verfügung stellen.
		\item RION muss eine Liste der auf dem Server vorliegenden Pakete herunterladen und durchsuchen können.
		\item RION muss Pakete installieren können. Dabei müssen auch ältere Versionen des Paketes zugänglich gemacht werden können.
		\item RION muss installierte Pakete aktualisieren können. In diesem Kontext bedeutet das, dass die neuere Version des Paketes parallel zur Alten zu installiert wird.
		\item RION muss installierte Pakete entfernen können.
\end{itemize}

\subsection{Interaktion}


\begin{itemize}
	\item Dem Benutzer muss ein CLI zur Verfügung stehen.
\end{itemize}

\section{Wunschkriterien}
\begin{itemize}
	\item INOR soll Pakete signieren.
	\item INOR soll Hashwerte von Paketen anfertigen und an RION übermitteln können.
	\item RION soll nach dem Herunterladen, aber vor der Installation, Pakete auf ihre Signatur, sowie auf ihren korrekten Hashwert überprüfen.

\end{itemize}
