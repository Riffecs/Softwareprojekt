\chapter{Testzenarien}
\begin{itemize}
	\item[T0110] \textit{Installation:}	Der Nutzer installiert mit \quotes{rion install \textit{Packagename}} ein Paket. Das Paket wird, inklusive eventueller Abhängigkeiten, installiert. Skripte (Post-Install-Prozessierung) werden erfolgreich ausgeführt.\\\\
	\textbf{Anforderung kann laut Blackboxtest nur teilweise erfüllt werden. Das Paket wird installier, es existiert aber keine Abhänigkeitsbehandlung}
	\item[T0120] \textit{Suchen:} Der Nutzer sucht mit \quotes{rion search \textit{text}} ein Paket und bekommt eine Liste von passenden Paketen, inklusive einer kurzen Beschreibung derer, ausgeben.
    
    \textbf{Anforderung wurde laut Blackboxtest erfüllt, wobei bei den Textpaketen keine Beschreibung vorliegt.}
    
	\item[T0130] \textit{Information:} Der Nutzer sucht mit \quotes{rion info \textit{Packagename}} Informationen zu einem Paket und bekommt diese detailliert zu einem Paket ausgeben.\\\\
	\textbf{Anforderung wurde noch nicht implementiert.}
	
	\item[T0140] \textit{Aktualisieren:} Der Nutzer aktualisiert mit \quotes{rion update} alle installierten Pakete. Diese werden aktualisiert. Er aktualisiert mit \quotes{rion upgrade \textit{Packagename}} ein bestimmtes Paket, welches nun aktualisiert wird.\\\\

	\item[T0150] \textit{Entfernen:} Der Nutzer entfernt mit \quotes{rion remove \textit{Packagename/s}} ein oder mehrere Pakete, welche dadurch deinstalliert werden.\\\\
	Anforderung wurde laut Blackboxtest erfüllt.
	\item[T0160] \textit{Anleitung:} Der Nutzer schreibt \quotes{man rion}, worauf ihm die Manpage zu RION angezeigt wird.\\\\
		Anforderung wurde laut Blackboxtest erfüllt.
	\item[T0170] \textit{Installierte Pakete listen:} Der Nutzer gibt \quotes{\textit{rion list (package\_name\_pattern)}} ein. Er erhält Liste aller installierten Pakete, bzw. aller installierten Funktionen eines Paketes.\\\\
		Anforderung wurde noch nicht implementiert.
	\item[T0180] Die Repositories, auf die RION zugreift, können mit einem config file angepasst werden.\\\\
		Anforderung wurde laut Blackboxtest erfüllt.
	\item[T0190] RION kann mehrere virtuellen Umgebungen verwalten.\\\\
		Anforderung wurde laut Blackboxtest erfüllt.
	\item[T0111] RION kann mit 	\quotes{rion check (\textit{Packagename (Packageversion)})} überprüfen, ob bestimmte oder alle installierten Pakete noch korrekt installiert sind.\\\\
		Anforderung wurde noch nich erfüllt.
	\item[T0121] RION kann mit \quotes{rion update} die lokale Datenbank aktualisieren.\\\\
		Anforderung wurde laut Blackboxtest erfüllt.
	\item[T0131] RION kann sich mit \quotes{rion auth -u >>USER<< -p >>PWD<<} anmelden. RION kann sich auch über einen config file (nur für den Ersteller lesbar) anmelden. Sollte keine Anmeldung vorliegen, fragt RION beim ersten Aufruf nach den Anmeldedaten.\\\\
	Anforderung wurde laut Blackboxtest erfüllt. Allerdings wurde die Syntax leicht angepasst.
\end{itemize}
