\chapter{Kritische Betrachtung}

Abschließend ergaben sich aus der Arbeit am Projekt „RION“ vielschichtige Erkenntnisse, die retrospektiv betrachtet werden müssen.
Idealer Weise sollten dem Projekt gewisse Parameter in der Vorgehensweise vorausgesetzt sein, die für den Erfolg der Durchführung unseres Erachtens unabdingbar sind. Dazu gehören als Mindestvoraussetzung die termingerechte Gruppenbildung und eine grundsätzliche Themen-Affinität der Gruppenmitglieder. Einerseits, um den vorgeschriebenen Zeitrahmen optimal nutzen zu können und andererseits das Potenzial an Motivation und Kompetenzen in der höchsten Grade ausschöpfen zu können. Also braucht es eine gewisse sinnhafte Kompatibilität des Teams untereinander. Hierzu zählt jedenfalls ein annähernd gleicher Wissensstand bei grundlegenden projektimmanenten Inhalten. Je weniger fachlicher Aufholbedarf untereinander ausgeglichen werden muss, desto effizienter der Fortschritt der Gemeinschaftsarbeit. Hierbei wird das gegenseitige profitieren und dazulernen auf fachfremden Gebieten der Team-Mitglieder aus unterschiedlichen Bereichen nicht grundsätzlich in Abrede gestellt. Wohlgemerkt gilt die Kritik einem mangelnden gemeinsamen Grundwissen. Fachlich betrachtet meint dies bei vorliegendem Projekt, dass sich alle Teilnehmer mit der jeweils angewandten Programmiersprache, der UML, dem Ziel-Betriebssystem, dem genutzten Vorgehensmodell und sowie grundlegenden Programmierparadigmen wie der OOP auskennen müssten.
Von größter Bedeutung für Erfolg oder Misserfolg des Projektes sind auch soziale Komponenten. Eine unbedingte Kommunikations- und Kooperations-Bereitschaft muss zwingend kleinster gemeinsamer Nenner im Team sein. Selbst fruchtbringende Alleingänge müssen dem Team transparent und diskutierbar gemacht werden. Teamwork muss Priorität haben, um gemeinsame Fortschritte zu generieren. Dazu bedarf es des unumstößlichen Willens eigene Fähigkeiten in den Dienst des Teams zu stellen und sich nicht nur selbst vordergründig profilieren zu wollen. Dass das Team nach außen hin als solches auftritt, muss dabei selbstverständlich sein.
Zur Organisation der Teilnehmer ist es zwingend angeraten, dass zentrale Plattformen gewählt, bedient und regelmäßig und zuverlässig besucht werden – eine Grundkomponente effektiver Kommunikation!
Ideal sollte schließlich das Ergebnis zielorientiert zu den gewünschten Ergebnissen führen.\\
In erster Linie müssen die Kundenanforderungen bedarfsgerecht erfüllt worden sein. Das heißt, das Ergebnis des Projektes muss voll funktionsfähig sein, was in diesem Fall scheiterte. Wünschenswert waren auch mögliche zusätzlich eingebrachte Funktionen, die im Verlauf des Projektes ebenfalls angedacht worden waren.
Termintreu hätten abschließend alle Tests bestanden werden müssen und non-funktionale Anforderungen übertroffen werden können. Damit sind zum Beispiel Fragen von Sicherheit, Leistungsfähigkeit, Robustheit u.a. gemeint.
In vorliegendem Fall konnten viele dieser Ansprüche aus verschiedensten Gründen nicht realisiert werden, unter anderem deshalb, da voraussetzende Parameter nicht gegeben waren.

