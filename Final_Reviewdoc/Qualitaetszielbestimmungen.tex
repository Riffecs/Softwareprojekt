\chapter{Qualitätszielbestimmung}

\begin{itemize}
	\item Robustheit: wichtig
	\item Zuverlässigkeit: sehr wichtig
	\item Korrektheit: sehr wichtig
	\item Benutzerfreundlichkeit: sehr wichtig
	\item Effizienz: wichtig
	\item Echtzeitfähigkeit: weniger wichtig
	\item Portierbarkeit: weniger wichtig
	\item Kompatibilität: wichtig
\end{itemize}

Der Grund für die Existenz von RION ist der hohe manuelle Aufwand der vorigen Lösungen.
Daher hat Benutzerfreundlichkeit für RION hohe Priorität.
Doch dies allein kann für einen Package-Manager nicht genügen, da sowohl die zuverlässige, als auch korrekte Installation von Paketen Benutzerfreundlichkeit erst ermöglicht. Darüber hinaus kann nur durch ausreichende Robustheit die Verwendung im Alltag garantiert werden. Diese Robustheit muss angesichts der großen zu transferierenden Datenmengen durch Effizienz komplementiert werden.
Kompatibilität ist in soweit wichtig, als dass RION unter einer Vielzahl unterschiedlicher GNU/Linux-Distributionen laufen soll. Da GNU/Linux aber das einzige Zielsystem ist, ist Portierbarkeit weniger wichtig, obgleich dennoch durch Python gewährleistet.
