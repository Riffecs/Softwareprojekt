	%% ++++++++++++++++++++++++++++++++++++++++++++++++++++++++++++
%% Hauptdatei, Wurzel des Dokuments
%% ++++++++++++++++++++++++++++++++++++++++++++++++++++++++++++

% Headerfeld, Typ des Dokumentes, einzubindende Packages.
% Hier bei Bedarf Änderungen vornehmen.
\documentclass
[   twoside=false,     % Einseitiger oder zweiseitiger Druck?
    fontsize=12pt,     % Bezug: 12-Punkt Schriftgröße
    DIV=15,            % Randaufteilung, siehe Dokumentation "KOMA"-Script
    BCOR=17mm,         % Bindekorrektur: Innen 17mm Platz lassen. Copyshop-getestet.
%    headsepline,
    headsepline,  % Unter Kopfzeile Trennlinie (aus: headnosepline)
    footsepline,  % Über Fußzeile Trennlinie (aus: footnosepline)
    open=right,        % Neue Kapitel im zweiseitigen Druck rechts beginnen lassen
    paper=a4,          % Seitenformat A4
    abstract=true,     % Abstract einbinden
    listof=totoc,      % Div. Verzeichnisse ins Inhaltsverzeichnis aufnehmen
    bibliography=totoc,% Literaturverzeichnis ins Inhaltsverzeichnis aufnehmen
    titlepage,         % Titelseite aktivieren
    headinclude=true,  % Seiten-Head in die Satzspiegelberechnung mit einbeziehen
    footinclude=false, % Seiten-Foot nicht in die Satzspiegelberechnung mit einbeziehen
    numbers=noenddot   % Gliederungsnummern ohne abschließenden Punkt darstellen
]   {scrreprt}         % Dokumentenstil: "Report" aus dem KOMA-Skript-Paket

\usepackage[active]{srcltx}
\overfullrule=2cm
%\usepackage[activate=normal]{pdfcprot} % Optischer Randausgleich -> pdflatex!
\usepackage{ifthen}
\usepackage[ngerman]{babel}   % Neue Deutsche Rechtschreibung
%\usepackage[latin1]{inputenc} % Zeichencodierung nach ISO-8859-1
\usepackage[utf8]{inputenc}   %	Zeichencodierung nach UTF-8 (Unicode)
\usepackage[T1]{fontenc}
\usepackage{graphicx}
%\usepackage{ae} % obsolet und durch lmodern ersetzt
\usepackage{lmodern}
\usepackage{listings}
\usepackage[T1]{url}
\usepackage{amsthm}
\usepackage{amsmath}
\usepackage{graphicx}
\RequirePackage{scrlfile}
\ReplacePackage{scrpage2}{scrlayer-scrpage}
% old: \usepackage[automark]{scrpage2}
\usepackage[automark]{scrlayer-scrpage}
\usepackage{setspace}
%\usepackage[first,light]{draftcopy} % Für Probedruck
\usepackage[plainpages=false,pdfpagelabels,hypertexnames=false]{hyperref}



%% UNIX 
\makeatletter
\DeclareOldFontCommand{\rm}{\normalfont\rmfamily}{\mathrm}
\DeclareOldFontCommand{\sf}{\normalfont\sffamily}{\mathsf}
\DeclareOldFontCommand{\tt}{\normalfont\ttfamily}{\mathtt}
\DeclareOldFontCommand{\bf}{\normalfont\bfseries}{\mathbf}
\DeclareOldFontCommand{\it}{\normalfont\itshape}{\mathit}
\DeclareOldFontCommand{\sl}{\normalfont\slshape}{\@nomath\sl}
\DeclareOldFontCommand{\sc}{\normalfont\scshape}{\@nomath\sc}
\makeatother

%% 

% Tiefe der Kapitelnummerierung beeinflussen
\setcounter{secnumdepth}{3} % Tiefe der Nummerierung
\setcounter{tocdepth}{3}    % Tiefe des Inhaltsverzeichnisses

% Datum anpassen
\newcommand{\leadingzero}[1]{\ifnum #1<10 0\the#1\else\the#1\fi}
\renewcommand{\today}{\leadingzero{\day}.\leadingzero{\month}.\the\year}     % DD.MM.YYYY

% Hier in die zweite geschweifte Klammer jeweils
% die persoenlichen Daten und das Thema der Arbeit eintragen:
\newcommand{\artderausarbeitung}{Pflichtenheft}
\newcommand{\namedesautorsI}{P.~Augustin, J.~Skopp, C.~Juin, }
\newcommand{\namedesautorsII}{G.~Lehmann, A.~Schmidt, R.~Schöne}
\newcommand{\themaderarbeit}{RION  Package-Manager}
\newcommand{\xRION}{\RION}
%% Das passt schon so 
\newcommand{\namedesautors}{\namedesautorsI \namedesautorsII}

% PDF Metadaten definieren
\hypersetup{
   pdftitle={\themaderarbeit},
   pdfsubject={\artderausarbeitung},
   pdfauthor={\namedesautors},
   pdfkeywords={\artderausarbeitung; TU-Ilmenau}}


% Abkürzungsverzeichnis beeinflussen. Hier nichts ändern!
\usepackage[intoc]{nomencl}
  \AtBeginDocument{\setlength{\nomlabelwidth}{.25\columnwidth}}
  \let\abbrev\nomenclature
  \renewcommand{\nomname}{Abkürzungsverzeichnis und Formelzeichen}
  \renewcommand{\nomlabel}[1]{#1 \dotfill}
  \setlength{\nomitemsep}{-\parsep}
  \makenomenclature

\usepackage[normalem]{ulem}
  \newcommand{\markup}[1]{\textbf{#1}}

% Seitenlayout festlegen. Hier nichts ändern!
\pagestyle{scrplain}
\ihead[]{\headmark}
\ohead[]{\pagemark}
\chead[]{}
\ifoot[]{}
\ofoot[]{\scriptsize \artderausarbeitung\ - \namedesautors}
\cfoot[]{}
\renewcommand{\titlepagestyle}{scrheadings}
\renewcommand{\partpagestyle}{scrheadings}
\renewcommand{\chapterpagestyle}{scrheadings}
\renewcommand{\indexpagestyle}{scrheadings}



% Abschnittsweise Nummerierung anstatt fortlaufend. Hier nichts ändern!
\makeatletter
\@addtoreset{equation}{chapter}
\@addtoreset{figure}{chapter}
\@addtoreset{table}{chapter}
\renewcommand\theequation{\thechapter.\@arabic\c@equation}
\renewcommand\thefigure{\thechapter.\@arabic\c@figure}
\renewcommand\thetable{\thechapter.\@arabic\c@table}
\makeatother
\renewcommand*{\pagedeclaration}[1]{\unskip, \hyperpage{#1}}

% Quelltextrahmen, klein. Hier nichts ändern!
\newsavebox{\inhaltkl}
\def\rahmenkl{\sbox{\inhaltkl}\bgroup\small\renewcommand{\baselinestretch}{1}\vbox\bgroup\hsize\textwidth}
\def\endrahmenkl{\par\vskip-\lastskip\egroup\egroup\fboxsep3mm%
\framebox[\textwidth][l]{\usebox{\inhaltkl}}}

% Quelltextrahmen, normale Groesse. Hier nichts ändern!
\newsavebox{\inhalt}
\def\rahmen{\sbox{\inhalt}\bgroup\renewcommand{\baselinestretch}{1}\vbox\bgroup\hsize\textwidth}
\def\endrahmen{\par\vskip-\lastskip\egroup\egroup\fboxsep3mm%
\framebox[\textwidth][l]{\usebox{\inhalt}}}


% Sonstige Befehlsdefinitionen hier ablegen.
\newcommand{\entspricht}{\stackrel{\wedge}{=}}
\newcommand{\quotes}[1]{\glqq#1\grqq{}}
\newcommand{\x}{X-FAB} % sry :)
\newcommand{\e}{RION} % well
\newcommand{\Linux}{GNU/Linux}
\makenomenclature


% Tabellenspaltendefinitionen mit fester Breite --> somit Zeilenumbruch innerhalb einer Zelle möglich
% aus http://www.torsten-schuetze.de/tex/tabsatz-2004.pdf
\usepackage{array, booktabs}
\newcolumntype{f}{>{$}l<{$}}
\newcolumntype{n}{>{\raggedright}l}
\newcolumntype{N}{>{\scriptsize}l}
\newcolumntype{v}[1]{>{\raggedright\hspace{0pt}}m{#1}}
\newcolumntype{V}[1]{>{\scriptsize\raggedright\hspace{0pt}}m{#1}}
\newcolumntype{Z}[1]{>{\raggedright\centering}m{#1}}
\newcolumntype{k}[1]{>{\raggedright}p{#1}}
% ergibt Tabllenspalte fester Breite, linksbündig
% Umbruch innerhalb der Zelle mit \\, neue Tabellezeile mit \tabularnewline
% \addlinespace für Gruppentrennung (aus \texttt{booktabs.sty})


\begin{document}
\onehalfspacing

\begin{titlepage}
	\centering
	{\Large \textsc{Technische Universität Ilmenau}}\\[3ex]
	{\Large Rechnerarchitekturen und Eingebettete Systeme}\\[3ex]
	\vfill
	{\Large \textbf{\artderausarbeitung}}\\[4ex]
	{\large \textbf{\themaderarbeit}}\\[5ex]
	%{\large \textbf{\xRION}}\\[5ex]
	\vfill
	\begin{tabular}{rl}
		\hline\\
		vorgelegt von:          & \quad \namedesautorsI\\[1,5ex]
										       & \quad \namedesautorsII\\[1,5ex]
		eingereicht am:         & \quad 
		\today \\[1,5ex]
		Fachgebiet:            & \quad Rechnerarchitekturen und Eingebettete Systeme\\[0,5ex]
								& \quad Institut für Mikroelektronik- und Mechatronik-System\\[1,5ex]
		Betreuer:            	& \quad Georg Gläser, Andreas Becher \\[1,5ex]
		Seminarleiter           & \quad Prof. Armin Zimmermann \\[1,5ex]	
	\end{tabular}
	\vfill
	
    


\end{titlepage}
%%% ++++++++++++++++++++++++++++++++++++++++++++++++++++++++++++
%% Zusammenfassung, Abstract
%% ++++++++++++++++++++++++++++++++++++++++++++++++++++++++++++


\renewcommand{\abstractname}{Kurzfassung}
\begin{abstract}
	\begin{center}
		X-FAB bietet eine Fülle an unterschiedlichen Technologien für diverse und auch
        spezifische Anwendungsmärkte an. Um Packete für das Process Design Kit verwalten zu können soll ein Package-Manager mit dem Namen RION geschaffen werden
	\end{center}
\end{abstract}

% Inhaltsverzeichnis
\cleardoublepage % Seitenumbruch erzwingen vor Änderung des Nummerierungsstils
\pagenumbering{roman} % Nummerierung der Seiten ab hier: i, ii, iii, iv...
\pagestyle{scrheadings} % Ab hier mit Kopf- und Fusszeile
\tableofcontents

% Die einzelnen Kapitel
\cleardoublepage % Seitenumbruch erzwingen vor Änderung des Nummerierungsstils
\pagenumbering{arabic} % Nummerierung der Seiten ab hier: 1, 2, 3, 4...

%%% Content %%%
\part*{Pflichtenheft}

%% Pages
\chapter{Zielbestimmungen}

RION ist ein Package-Manager zum suchen, installieren, aktualisieren und verwalten von Packages für das Process Design Kit (PDK). Die Pakete für RION werden von der Serverseite INOR verwaltet.

\section{Problemstellung}
Zur Zeit werden Packages manuell von einem Webinterface heruntergeladen. Sie werden manuell installiert und aktualisiert. Dieser jetzige Weg ist jedoch unübersichtlich, fehleranfällig, inkonsistent, kostet sehr viel Zeit und erfordert sehr viel manuelle Arbeit, sowohl server- als auch clientseitig. Daher soll dieser Prozess durch Verwendung von RION, clientseitig für X-Fab-Kunden, und INOR, serverseitig für X-FAB-Entwickler, vereinfacht und automatisiert werden.

\section{Musskriterien}
\subsection{Funktionalität}
\begin{itemize}
		\item INOR muss die Packages serverseitig über eine Datenbank in Form von Archiven zur Verfügung stellen.
		\item RION muss eine Liste der auf dem Server vorliegenden Pakete herunterladen und durchsuchen können.
		\item RION muss Pakete installieren können. Dabei müssen auch ältere Versionen des Paketes zugänglich gemacht werden können.
		\item RION muss installierte Pakete aktualisieren können. In diesem Kontext bedeutet das, dass die neuere Version des Paketes parallel zur Alten zu installiert wird.
		\item RION muss installierte Pakete entfernen können.
\end{itemize}

\subsection{Interaktion}


\begin{itemize}
	\item Dem Benutzer muss ein CLI zur Verfügung stehen.
\end{itemize}

\section{Wunschkriterien}
\begin{itemize}
	\item INOR soll Pakete signieren.
	\item INOR soll Hashwerte von Paketen anfertigen und an RION übermitteln können.
	\item RION soll nach dem Herunterladen, aber vor der Installation, Pakete auf ihre Signatur, sowie auf ihren korrekten Hashwert überprüfen.

\end{itemize}

\chapter{Vorgehensmodell}
\section{Motivation und Auswahl des Vorgehensmodells}

Als Vorgehensmodell haben wir uns aufgrund der interdisziplinären Zusammensetzung und Größe unseres Teams, sowie der Forderung nach einem agilen Vorgehensmodell seitens der X-FAB für Scrum entschieden.\\
Weiterhin ermöglicht Scrum durch iteratives Vorgehen flexibel auf Anforderungsänderungen zu reagieren und in jeder Iteration einen funktionsfähigen Prototyp fertigzustellen.\\

Scrum benötigt folgende Rollen innerhalb des Teams:
\begin{itemize}
	\item[Product Owner:] Durch ihn erfolgt eine kontinuierliche Qualitätssicherung der Prototypen. Des Weiteren führt er den Product-Backlog und dokumentiert somit den Gesamtfortschritt des Projekts.

	\item[Scrum Master:] Er moderiert interne Meetings, achtet auf die Einhaltung der Scrum-Methoden, hilft bei der Formulierung der Zielstellungen und unterstützt alle Mitglieder bei aufkommenden Problemen innerhalb ihrer Aufgaben.

	\item[Developer:]
	Sind für die Umsetzung der Sprint-Ziele verantwortlich und führen jeweils einen eigenen Sprint-Backlog, in welchem der individuelle Aufgabenfortschritt dokumentiert wird.

\end{itemize}

\section{Interne Gruppenorganisation}
Innerhalb unseres Teams fungiert Arndt Schmidt als Scrum-Master und Valentin Nakov als Product Owner. Als Developer sind Philip Augustin, Calvin Chong Chen Juin, Georg Leander Lehmann, Jonathan Skopp innerhalb des Teams tätig.\\
\clearpage
Projektphasen:
\begin{itemize}
	\item[Phase 1:] Organisation der Projektphasen;
	\item[Phase 2:] Beginn der Durchführung der Scrum-Iterationen und somit der Implementierung des Entwurfs inklusive Komponententests. Pro Sprint wird eine Woche angesetzt, die \quotes{Daily Meetings} sind auf Samstag und Dienstag datiert und werden online durchgeführt. Die Planung des kommenden Sprints sowie die Review des abgeschlossenen Sprints finden am Donnerstag statt.
	\item[Phase 3:] Durchführung von Integrations-, Blackbox- und Whitebox-Tests sowie Erstellen der finalen Reviewdokumente.


\end{itemize}
\clearpage
\section{Meilensteine}

\begin{itemize}
	\item
		Der Package-Manager RION soll über PyPi (pip) installiert werden können. Das Ziel ist hier, dass jede Linux-Distribution den Manager wie jedes andere bekannte \quotes{Command Line Programm} nutzen kann.

	\item Der Packagemanager RION soll sich erfolgreich \quotes{informieren} können, ob er alle erforderlichen Anforderungen erfüllt und gegebenenfalls interagieren kann. Dazu zählt Folgendes:

		\begin{itemize}
			\item Erstellen von Datenbanken auf INOR und dem lokalen System, um zu prüfen, ob die Packages aktuell sind
			\item Verwalten dieser Datenbanken
			\item Prüfen auf Korrektheit  des eigentlichen Package-Managers
			\item Prüfen auf Vollständigkeit des eigentlichen Package-Managers
			\item Erstellen und Verwalten von Sicherheitszertifikaten
		\end{itemize}
	\item Der Package-Manager RION kann Packages herunterladen, installieren, entpacken, verifizieren.
	\item Der Package-Manager RION kann prüfen (unter Zuhilfenahme der oben genannten Datenbank), ob es Aktualisierungen gibt.
	\item Der Package-Manager RION kann Abhängigkeiten erkennen und unter den oben genannten Methoden verwalten.
	\item Der Package-Manager RION kann ein Abbild aller installierten Packages inklusive der daraus resultierenden Abhängigkeiten erstellen können und diese über geeignete Wege ausgeben und auch wieder einlesen.
	\item Der Package-Manager RION ist fertiggestellt.
\end{itemize}

\chapter{Produkteinsatz}

\section{Anwendungsbereiche}
Der vorliegende Package-Manager richtet sich an Entwickler, sowie Kunden und Entwickler der \x. \\
RION soll die technischen Grundanforderungen zur PDK-Installation auf Kundenseite auf ein Minimum begrenzen. Durch entsprechende PDK-Paketinformationen und eine Plausibilitätsprüfung dieser, durch RION \quotes{(rion check)}, sollen Fehler bei der Installation von PDK-Paketen verhindert werden. Diese könnten im schlimmsten Fall zu einer Fehlfunktion des Halbleiterchips führen.
\section{Zielgruppe}
Die Zielgruppe umfasst alle Kunden der X-FAB.


\section{Betriebsbedingungen}
RION wird über pip installiert. Es läuft unabhängig von anderen \x-Systemen. Es ist vielmehr als Ergänzung zu sehen.\\
Der User muss mittels eines Befehls das Programm starten.\\
Der ausführenden Person muss bekannt sein, welches Paket installiert oder verwaltet werden soll.

\chapter{Produktumgebung}
Im folgenden wird aufgelistet, welche \quotes{elementaren Programme} auf dem Computer installiert sein müssen, damit \e~ normal funktioniert.\\

\section{Software}
Auf den verwendenten Rechnern sollte im Normalfall alle notwendige Software installiert sein, dennoch ist hier noch eine Übersicht zu finden:

\begin{itemize}
	\item Python $ \geq 3.6$
	\item RHEL 7 (bzw. Oracle Linux OL 7)
\end{itemize}

\section{Hardware}
\begin{itemize}
	\item genügend Speicherplatz
\end{itemize}

\chapter{Produktfunktionen}

\section{RION}

\begin{itemize}
	\item[F0110] \textit{Installation:} Der Nutzer kann mit \quotes{rion install \textit{Packagename}} Pakete, inklusive eventuellen Abhängigkeiten, installieren. Bei der Installation sollen Skripte (Post-Install-Prozessierung) ausgeführt werden können.
	\item[F0120] \textit{Suchen:} Der Nutzer kann mit \quotes{rion search \textit{text}} Pakete suchen und bekommt eine Liste von passenden Paketen, inklusive einer kurzen Beschreibung derer, ausgeben.
	\item[F0130] \textit{Information:} Der Nutzer bekommt mit \quotes{rion info \textit{Packagename}} detaillierte Informationen zu einem Paket ausgegeben.
	\item[F0140] \textit{Aktualisieren:} Der Nutzer kann mit \quotes{rion upgrade} alle installierten Pakete aktualisieren oder er kann mit \quotes{rion upgrade \textit{Packagename}} ein bestimmtes Paket aktualisieren.
	\item[F0150] \textit{Entfernen:} Der Nutzer kann mit \quotes{rion remove \textit{Packagename}} Pakete deinstallieren.
	\item[F0160] \textit{Anleitung:} Der Nutzer erhält durch \quotes{man rion} eine Manpage, in der alle wichtigen Funktionen von RION beschrieben werden.
	\item[F0170] \textit{Installierte Pakete listen:} Der Nutzer erhält durch \quotes{\textit{rion list (name)}} eine Liste aller installierten Pakete, bzw aller installierten Funktionen eines Paketes.
	\item[F0180] Die Repositories, auf die RION zugreift, soll mit einem config file angepasst werden.
	\item[F0190] RION soll mehrere virtuellen Umgebungen verwalten können.
	\item[F0111] RION soll mit 	\quotes{rion check (\textit{Packagename (Packageversion)})} überprüfen können, ob bestimmte oder alle installierten Pakete noch korrekt installiert sind.
	\item[F0121] RION soll mit \quotes{rion update} die lokale Datenbank aktualisieren.
	\item[F0131] RION kann sich mit \quotes{rion auth -u >>USER<< -p >>PWD<<} anmelden. RION kann sich auch über einen config file (nur für den Ersteller lesbar) anmelden. Sollte keine Anmeldung vorliegen, fragt RION beim ersten Aufruf nach den Anmeldedaten.
\end{itemize}

\section{INOR}

\begin{itemize}
	\item[F0210] \textit{Paket hinzufügen:} Sofern noch keine Version des Paketes in der Datenbank existiert, kann der Nutzer mit \quotes{inor add \textit{Packagename Packagefile}} ein Paket zur Datenbank hinzufügen. Abhängigkeiten werden automatisch berücksichtigt.
	\item[F0220] \textit{Neue Version hinzufügen:} Der Nutzer kann mit \quotes{inor update \textit{name version file}} eine neue Version eines Paketes zur Datenbank hinzufügen. Alte Versionen bleiben erhalten.
	\item[F0230] \textit{Beschreibung hinzufügen:} Der Nutzer kann mit \quotes{inor describ \textit{Packagename}} jene Beschreibung zu einem Paket hinzufügen, die beim Suchen durch RION abgerufen wird.
	\item[F0240] \textit{Lizenz:} Der Nutzer kann mit \quotes{rion license \textit{Packagename}} Paketen eine Lizenz hinzufügen.
	\item[F0250] \textit{Paket entfernen:} Der Nutzer kann mit \quotes{inor remove \textit{Packagename}} alle Versionen eines Paketes aus der Datenbank entfernen oder er kann mit \quotes{inor remove \textit{Packagename Versionsnummer}} eine bestimmte Version eines Paketes aus der Datenbank entfernen.
	\item[F0260] \textit{Paket signieren:} Der Nutzer kann mit \quotes{inor sign} alle installierten Pakete in der Datenbank signieren oder er kann mit \quotes{rion sign \textit{Packagename (Versionsnummer)}} ein bestimmtes Paket oder auch nur eine bestimmte Version eines Paketes signieren.
	\item[F0270] \textit{Pakete publizieren:} Der Nutzer kann mit \quotes{inor publish \textit{Packagename (Packageversion)}} alle spezifizierten Pakete für RION freischalten.
	\item[F0280] \textit{Pakete sperren:} Der Nutzer kann mit \quotes{\textit{inor unpublish Packagename (Packageversion)}} alle spezifizierten Pakete für RION sperren.
	\item[F0290] \textit{Anleitung:} Der Nutzer erhält durch \quotes{man inor} eine Manpage, in der alle wichtigen Funktionen von INOR beschrieben werden.


\end{itemize}

\chapter{Qualitätszielbestimmung}

\begin{itemize}
	\item Robustheit: wichtig
	\item Zuverlässigkeit: sehr wichtig
	\item Korrektheit: sehr wichtig
	\item Benutzerfreundlichkeit: sehr wichtig
	\item Effizienz: wichtig
	\item Echtzeitfähigkeit: weniger wichtig
	\item Portierbarkeit: weniger wichtig
	\item Kompatibilität: wichtig
\end{itemize}

Der Grund für die Existenz von RION ist der hohe manuelle Aufwand der vorigen Lösungen.
Daher hat Benutzerfreundlichkeit für RION hohe Priorität.
Doch dies allein kann für einen Package-Manager nicht genügen, da sowohl die zuverlässige, als auch korrekte Installation von Paketen Benutzerfreundlichkeit erst ermöglicht. Darüber hinaus kann nur durch ausreichende Robustheit die Verwendung im Alltag garantiert werden. Diese Robustheit muss angesichts der großen zu transferierenden Datenmengen durch Effizienz komplementiert werden.
Kompatibilität ist in soweit wichtig, als dass RION unter einer Vielzahl unterschiedlicher GNU/Linux-Distributionen laufen soll. Da GNU/Linux aber das einzige Zielsystem ist, ist Portierbarkeit weniger wichtig, obgleich dennoch durch Python gewährleistet.

\chapter{Testszenarien}

\section{RION}

\begin{itemize}
	\item[T0110] \textit{Installation:} Der Nutzer installiert mit \quotes{rion install \textit{Packagename}} ein Paket. Das Paket wird, inklusive eventueller Abhängigkeiten, installiert. Skripte (Post-Install-Prozessierung) werden erfolgreich ausgeführt.
	\item[T0120] \textit{Suchen:} Der Nutzer sucht mit \quotes{rion search \textit{text}} ein Paket und bekommt eine Liste von passenden Paketen, inklusive einer kurzen Beschreibung derer, ausgeben.
	\item[T0130] \textit{Information:} Der Nutzer sucht mit \quotes{rion info \textit{Packagename}} Informationen zu einem Paket und bekommt diese detailliert zu einem Paket ausgeben.
	\item[T0140] \textit{Aktualisieren:} Der Nutzer aktualisiert mit \quotes{rion update} alle installierten Pakete. Diese werden aktualisiert. Er aktualisiert mit \quotes{rion upgrade \textit{Packagename}} ein bestimmtes Paket, welches nun aktualisiert wird.
	\item[T0150] \textit{Entfernen:} Der Nutzer entfernt mit \quotes{rion remove \textit{Packagename/s}} ein oder mehrere Pakete, welche dadurch deinstalliert werden.
	\item[T0160] \textit{Anleitung:} Der Nutzer schreibt \quotes{man rion}, worauf ihm die Manpage zu RION angezeigt wird.
	\item[T0170] \textit{Installierte Pakete listen:} Der Nutzer gibt \quotes{\textit{rion list (package\_name\_pattern)}} ein. Er erhält Liste aller installierten Pakete, bzw. aller installierten Funktionen eines Paketes.
	\item[T0180] Die Repositories, auf die RION zugreift, können mit einem config file angepasst werden.
	\item[T0190] RION kann mehrere virtuellen Umgebungen verwalten.
	\item[T0111] RION kann mit 	\quotes{rion check (\textit{Packagename (Packageversion)})} überprüfen, ob bestimmte oder alle installierten Pakete noch korrekt installiert sind.
	\item[T0121] RION kann mit \quotes{rion update} die lokale Datenbank aktualisieren.
	\item[T0131] RION kann sich mit \quotes{rion auth -u >>USER<< -p >>PWD<<} anmelden. RION kann sich auch über einen config file (nur für den Ersteller lesbar) anmelden. Sollte keine Anmeldung vorliegen, fragt RION beim ersten Aufruf nach den Anmeldedaten.


\end{itemize}

\section{INOR}

\begin{itemize}
	\item[T0210] \textit{Paket hinzufügen:} Der Nutzer fügt mit \quotes{inor add \textit{Packagename Packagefile}} ein Paket zur Datenbank hinzu, welches noch nicht in der Datenbank ist. Daraufhin befindet es sich in der Datenbank. Abhängigkeiten wurden berücksichtigt.
	\item[T0220] \textit{Neue Version hinzufügen:} Der Nutzer fügt mit \quotes{inor update \textit{Packagename Packageversion Packagefile}} eine neue Version eines vorhandenen Paketes zur Datenbank hinzu. Es befindet sich daraufhin in der Datenbank.
	\item[T0230] \textit{Beschreibung hinzufügen:} Der Nutzer fügt mit \quotes{inor describ \textit{Packagename}} jene Beschreibung zu einem Paket hinzu, die beim Suchen durch RION abgerufen wird. Die Beschreibung ist nun abrufbar.
	\item[T0240] \textit{Lizenz:} Der Nutzer fügt mit \quotes{rion install \textit{Packagename}} Lizenzinformation hinzu.
	\item[T0250] \textit{Paket entfernen:} Der Nutzer entfernt mit \quotes{inor remove \textit{Packagename}} alle Versionen eines Paketes aus der Datenbank oder er kann mit \quotes{inor remove \textit{Packagename Versionsnummer}} eine bestimmte Version eines Paketes aus der Datenbank entfernen.
	\item[T0260] \textit{Paket signieren:} Der Nutzer signiert mit \quotes{inor sign} alle installierten Pakete in der Datenbank oder er signiert mit \quotes{rion sign \textit{Packagename (Versionsnummer)}} ein bestimmtes Paket oder auch nur eine bestimmte Version eines Paketes. Die spezifizierten Pakete sind daraufhin signiert.
	\item[T0270] \textit{Pakete publizieren:} Der Nutzer schaltet mit \quotes{textit{inor publish Packagename (Packageversion)}} alle spezifizierten Pakete für RION frei. Daraufhin kann RION auf die spezfizierten Pakete zugreifen.
	\item[T0280] \textit{Pakete publizieren:} Der Nutzer sperrt mit \quotes{\textit{inor unpublish Packagename (Packageversion)}} alle spezifizierten Pakete für RION. Daraufhin kann RION auf die spezfizierten Pakete nicht mehr zugreifen.
	\item[T0290] \textit{Anleitung:} Der Nutzer schreibt \quotes{man inor}, worauf ihm die Manpage zu RION angezeigt wird.
\end{itemize}

\chapter{Entwicklungsumgebungen}

\section{Software}

\begin{itemize}
	\item Plattform
		\begin{itemize}
			\item Python mit allen benötigten Libraries sowie der pypi Libraries.
		\end{itemize}

		\begin{itemize}
			\item Neovim
			\item VS-Code / VS-Codium
			\item TeXstudio
			\item \LaTeX
			\item Kommunikation via Jitsi, Signal, MSTeams, Discord
			\item Github, git
			\item draw.io
			\item Nextcloud
			\item \Linux~ (Fedora, Manjaro, Arch)
			\item Windows
		\end{itemize}
\end{itemize}

\section{Hardware}

\begin{itemize}
	\item Internetverbindung
	\item Computer und dazugehörige Peripheriegeräte
\end{itemize}

\chapter{UML}

\section{Weitere Sinnlose Diagramme}
\begin{minipage}{\linewidth}
\centering%
\includegraphics[width=0.8\linewidth,clip=]{./img/blödsinn.png}%
\captionof{figure}{blödsinn}%
\label{fig:blödsinn}%
\end{minipage}
\vspace{10px}

\section{Doxygen}

Die Ausführliche von Doxygen generierte Dokumentation ist im Anhang zu finden.





%% Syntax
% This file is a ghost. Only abbreviations for foreign words or new words are stored here. Nothing more
\makenomenclature
\nomenclature{RION}{Packagemanager und Schnittstelle für die Paketverwaltung zwischen X-FAB-Server und Client}

\nomenclature{X-FAB}{Die X-FAB ist ein Anbieter für Halbleitertechnologien, welcher sowohl die Fertigung als auch Designunterstützung für Kunden anbietet, die gemischt analog-digitale integrierte Schaltkreise (ICs) entwickeln. Die X-FAB bietet eine Reihe an unterschiedlichen Technologien für diverse und auch
spezifische Anwendungsmärkte an.}

\nomenclature{INOR}{Serverseitiges BackEnd bei der X-FAB.}

\nomenclature{Pakete/Packages}{Datenbündel, die als solches eine Funktion haben. Hiermit sind die Pakete auf den X-FAB-Servern gemeint.}

\nomenclature{pyftpdlib}{FTP-Server-Implementation für Python}

\nomenclature{Python}{Programmiersprache: \url{https://www.python.org/}}

\nomenclature{CLI}{Ein \texttt{C}ommand \texttt{L}ine \texttt{I}nterface ist eine Schnittstelle, die es dem Nutzer erlaubt, Textbefehle an ein Programm zu übermitteln. Betriebssysteme, darunter GNU/Linux stellen hierfür eine sogenannte Shell, wie zum Beispiel Bash, Dash oder ZSH, zur Verfügung, auf welche man mit einem Terminal zugreifen kann.}

\nomenclature{Manpage}{Eine Hilfeseite für installierte Programme unter Unix-artigen Systemen, auf der für gewöhnlich Befehle, Flags und Hinweise zur Benutzung eines Programms verzeichnet sind. Darauf zugegriffen wird mit \texttt{man >>name<<}.}

\nomenclature{pip}{Python Package-Manager: \url{https://pypi.org/project/pip/}}

\nomenclature{PDK}{Ein PDK ist eine komplexe Sammlung von technologiespezifischen Daten für den aufwändigen und
relativ teuren Entwurf von anwendungsspezifischen integrierten Schaltkreisen.}

\nomenclature{PyPi}{Pypi ist eine Softwaresammlung, durch die Pythonpackages via pip heruntergeladen werden können.}


%%% Ende %%%

\appendix
\part*{Anhang}


% Anahng
%% ++++++++++++++++++++++++++++++++++++++++++++++++++++++++++++
%% Anhang: Literaturverzeichnis
%% ++++++++++++++++++++++++++++++++++++++++++++++++++++++++++++


% Mit dem Befehl \nocite werden auch nicht im Text zitierte
% aus der Literaturdatenbank mit in das Literaturverzeichnis aufgenommen.
% Ein "\nocite{*}" übernimmt ungeprüft die komplette Datenbank.
%\nocite{*}

\cleardoublepage
\nocite{*}
\ihead[]{Literaturverzeichnis}
\bibliographystyle{acm}
\bibliography{literatur} % "literatur.bib" ist hier die einzige Literaturdatenbank.

% Alternativ: Mehrere Datenbanken verwenden, falls eine
% oder mehrere umfangreiche Sammlungen exisitieren:
%\bibliography{literatur_buecher,literatur_weblinks}


%% ++++++++++++++++++++++++++++++++++++++++++++++++++++++++++++
%% Anhang: Abbildungsverzeichnis
%% ++++++++++++++++++++++++++++++++++++++++++++++++++++++++++++


% Keine Änderungen vornehmen!
\cleardoublepage
\ihead[]{Abbildungsverzeichnis}
\listoffigures


%% ++++++++++++++++++++++++++++++++++++++++++++++++++++++++++++
%% Anhang: Abkürzungsverzeichnis
%% ++++++++++++++++++++++++++++++++++++++++++++++++++++++++++++


% Hier keine weiteren Änderungen vornehmen
\cleardoublepage
\ihead[]{Abkürzungsverzeichnis und Formelzeichen}
\printnomenclature{} % TODO: Hyperref




\end{document}
