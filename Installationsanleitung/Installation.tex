\documentclass[fleqn,10pt]{olplainarticle}
\usepackage{listings}
% Use option lineno for line numbers 

\title{Installationsanleitung - RION}

\author[1]{Jonathan Skopp}
\affil[1]{jonathan-erik.skopp@tu-ilmenau.de}

\keywords{RION, Python, pip, git}

\begin{abstract}
Dies ist eine kurze Installationsanleitung für RION
\end{abstract}

\begin{document}
\flushbottom
\maketitle
\thispagestyle{empty}

\section{Vorraussetzungen}
\begin{itemize}
    \item Python 3.8
    \item pip 
    \item setuptools
\end{itemize}

\section{Installation}
Auch wenn das grundlegende System bekannt ist, zeige ich hier nocheinmal eine Anleitung.

\section{Installation von Python 3.8}
\begin{listings}
sudo apt-get -y install python3.8
\end{listings}

\section{Installation von pip}
\begin{listings}
sudo apt-get -y install python3-pip
\end{listings}

\section{Installation von setuptools}
\begin{listings}
pip install setuptools
\end{listings}

\section{Installation des RION Packages}
\subsection{Installation von RION als Package}
\begin{listings}
pip install rion
\end{listings}
\subsection{Installation einer alten Version}
\begin{listings}
pip install rion==x.x.x
\end{listings}
Die Versionsinformationen entnehmen Sie bitte der Pypi\footnote{\url{https://pypi.org/project/rion/#history}}.
\subsection{Installation von Pre-Release Versionen}
\begin{listings}
pip install git+https://github.com/Riffecs/rion.git#egg=rion

\end{listings}


\section{Installation von rion}
\begin{listings}
rion installer
\end{listings}


\end{document}