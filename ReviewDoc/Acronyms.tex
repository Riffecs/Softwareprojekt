% This file is a ghost. Only abbreviations for foreign words or new words are stored here. Nothing more
\makenomenclature
\nomenclature{RION}{Packagemanager und Schnittstelle für die Paketverwaltung zwischen X-FAB-Server und Client}

\nomenclature{X-FAB}{Die X-FAB ist ein Anbieter für Halbleitertechnologien, welcher sowohl die Fertigung als auch Designunterstützung für Kunden anbietet, die gemischt analog-digitale integrierte Schaltkreise (ICs) entwickeln. Die X-FAB bietet eine Reihe an unterschiedlichen Technologien für diverse und auch
spezifische Anwendungsmärkte an.}

\nomenclature{INOR}{Serverseitiges BackEnd bei der X-FAB.}

\nomenclature{Pakete/Packages}{Datenbündel, die als solches eine Funktion haben. Hiermit sind die Pakete auf den X-FAB-Servern gemeint.}

\nomenclature{pyftpdlib}{FTP-Server-Implementation für Python}

\nomenclature{Python}{Programmiersprache: \url{https://www.python.org/}}

\nomenclature{CLI}{Ein \texttt{C}ommand \texttt{L}ine \texttt{I}nterface ist eine Schnittstelle, die es dem Nutzer erlaubt, Textbefehle an ein Programm zu übermitteln. Betriebssysteme, darunter GNU/Linux stellen hierfür eine sogenannte Shell, wie zum Beispiel Bash, Dash oder ZSH, zur Verfügung, auf welche man mit einem Terminal zugreifen kann.}

\nomenclature{Manpage}{Eine Hilfeseite für installierte Programme unter Unix-artigen Systemen, auf der für gewöhnlich Befehle, Flags und Hinweise zur Benutzung eines Programms verzeichnet sind. Darauf zugegriffen wird mit \texttt{man >>name<<}.}

\nomenclature{pip}{Python Package-Manager: \url{https://pypi.org/project/pip/}}

\nomenclature{PDK}{Ein PDK ist eine komplexe Sammlung von technologiespezifischen Daten für den aufwändigen und
relativ teuren Entwurf von anwendungsspezifischen integrierten Schaltkreisen.}

\nomenclature{PyPi}{Pypi ist eine Softwaresammlung, durch die Pythonpackages via pip heruntergeladen werden können.}
