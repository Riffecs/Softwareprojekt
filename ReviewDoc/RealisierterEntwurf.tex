\chapter{Realisierter Entwurf}
\section{altes Klassendiagramm}
[\intextsep]
\begin{minipage}{\linewidth}
\centering%
\includegraphics[width=0.8\linewidth,clip=]{./img/Img2.jpg}%
\label{fig:altes_Klassendiagramm}%
\end{minipage}
[\intextsep]


\section{neues Klassendiagramm}
[\intextsep]
\begin{minipage}{\linewidth}
\centering%
\includegraphics[width=0.8\linewidth,clip=]{./img/Img3.jpg}%
\label{fig:altes_Klassendiagramm}%
\end{minipage}
[\intextsep]

In der Implementierungsphase wurde die Erstellung eines neuen Klassendiagramms, aufgrund von Unzulänglichkeiten des vorigen, notwendig.
    
    
\section{Sequenzdiagramm}
[\intextsep]
\begin{minipage}{\linewidth}
\centering%
\includegraphics[width=0.8\linewidth,clip=]{./img/Img4.jpg}%
\label{fig:altes_Klassendiagramm}%
\end{minipage}
[\intextsep]
\clearpage

Entscheidung 1\\
Aufgabe: Protokoll, mit dem RION auf die Pakete des Servers zugreift\\

Ursprünglich war geplant, RION und INOR via Client-Server-Modell eng miteinader zu
verbinden. In der eigentlichen Implemetierung stellte sich jedoch heraus, dass ein
klassischer Ansatz sinnvoller ist. \\

Inor arbeitet unabhängig von RION. Seine Aufgabe besteht allein darin, Ordner mit Paketen
bzw. physischen Links auf Pakete zu befüllen. Auf diese wird dann mittels eines Protokolls
zugegriffen. Hierfür gibt es primär zwei verschlüsselte Optionen:\\

\begin{itemize}
    \item Option 1: https
    \begin{itemize}
        \item Vorteile:
            \begin{itemize}
                \item beste Übertragungsgeschwindigkeit
            \end{itemize}
        \item Nachteile:
            \begin{itemize}
                \item Arbeitet nicht direkt auf Dateienebene
                \item Aufwendige Implementierung der Authentifikation
            \end{itemize}
    \end{itemize}
     \item Option 2: ftps
    \begin{itemize}
        \item Vorteile
        \begin{itemize}
        \item Arbeitet direkt auf der Dateienebene
        \item pyftpflib bietet bereits eine ausgereifte und gut getestete Implemtierung zur
Authentifikation in Form von virtuellen Nutzern
        \end{itemize}
        \item Nachteile
        \begin{itemize}
            \item Nicht die beste, aber immer noch eine sehr gute Geschwindigkeit
        \end{itemize}
        Konklusion: Aufgrund der Möglichkeit virtuelle Nutzer anzulegen, ist ftps aus unserer Sicht
die für RION/INOR richtige Entscheidung.


        
    \end{itemize}
\end{itemize}
\clearpage

Entscheidung 2 \\
Aufgabe: Festlegen eines Tools zur Verwaltung von installierbaren und installierten Paketen.

\begin{itemize}
    \item Option 1: Sqlite3
    \begin{itemize}
        \item Vorteile
        \begin{itemize}
            \item native Einbindung
            \item komplexe Anfragen funktionieren einfacher
            \item leichtgewichtig
        \end{itemize}
        \item Nachteile
                
        \begin{itemize}
            \item Nicht ohne entsprechendes Programm lesbar
            \item Entsprechende Binarys für jede Plattform
        \end{itemize}
    \end{itemize}
     \item Option 2: json  
    \begin{itemize}
        \item Vorteile
        \begin{itemize}
            \item leichtgewichtiger
            \item plain text
        \end{itemize}
        \item Nachteile
        \begin{itemize}
            \item unübersichtlich bei großen Datenmengen
        \end{itemize}        
    \end{itemize}
\end{itemize}

Konklusion: Sqlite3 mag den Nachteil haben, dass es normalerweise eine systemspezifische
Binary benötigt, das wird jedoch durch die Einbindung in Python umgangen. Dies, kombiniert
mit den einfachen Möglichkeiten für komplexe Anfragen, hat den Ausschlag für Sqlite3
gegeben.

    