\chapter{Einleitung}
\section{Zweck des Systems}
Aufgabe ist die Erstellung eines Packagemanagers zur Suche, Installation, Aktualisierung
und Verwaltung von PDKs. 

\section{Entwurfsziele}
Der Packagemanager sollte in Python geschrieben werden.\\

Zuzüglich der oben genannten Ziele, sollte der Packagemanager auch das Management von
Abhängigkeiten, das Verwalten von virtuellen Umgebungen, sowie das Erkennen von bereits
installierten Paketen ermöglichen. Zudem sollten auch zu jedem Paket ein Informationstext
in Form von bereits vorliegenden Textdateien angezeigt werden können.\\


Serverseitig soll ein Programm die verfügbaren Pakete verwalten können. In einer
Datenbank, sollen der Name des Paketes, noch festzulegende Metadaten und
Abhängigkeiten gespeichert sein. Außerdem sollte der Server natürlich, dem
Packagemanager die Pakete selbst bereit stellen. \\

Client und Server zusammen sollen desweiteren über eine Rechteverwaltung bzw.
Authentifizierung verfügen, welche es verschieden Nutzern ermöglicht auf verschiedene
Pakete zuzugreifen.

\clearpage
\section{Überblick}
Implementiert wurden zum jetzigen Stand folgende Punkte:


\begin{itemize}
    \item  Entfernen von Paketen
    \item Suchen von Paketen
    \item Nutzerauthentifizierung
    \item Übertragung von Paketen vom Server zum Client
   \item Verwalten von virtuellen Umgebungen
   \item Installation von Paketen
 \item Konfigurationdatei
\end{itemize}

Noch nicht Implementiert sind:
\begin{itemize}
    \item Abhängigkeitenverwaltung
     \item Programm zur Verwaltung der auf dem Server verfügbaren Paketen
\end{itemize}